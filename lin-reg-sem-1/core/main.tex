\documentclass[12pt]{article}
\usepackage{statproj}

\title{Statistical Analysis of the Relationship Between Website Page Size and Load Time}
\author{Avigyan Chakraborty (BS2516)}
\date{November, 2025}

\begin{document}

% First page with title, name, date, and abstract
\maketitlepage{\thetitle}{\theauthor}{\thedate}{
% Your abstract content will go here
This is where you'll write your abstract content later. You can replace this text with your actual abstract when you're ready. The abstract should provide a brief overview of your statistics assignment, including the main objectives, methods, and key findings.

You can write multiple paragraphs here. The formatting will handle the spacing and make it look professional and well-organized for your statistics assignment.
}

% Second page with table of contents (clickable links and improved spacing)
\makecontents

% Third page - start your project here
\section{Introduction}
Your content starts here. This section will introduce your statistical analysis project and its significance.
Your content starts here. This section will introduce your statistical analysis project and its significance.
Your content starts here. This section will introduce your statistical analysis project and its significance.
Your content starts here. This section will introduce your statistical analysis project and its significance.
Your content starts here. This section will introduce your statistical analysis project and its significance.
Your content starts here. This section will introduce your statistical analysis project and its significance.
Your content starts here. This section will introduce your statistical analysis project and its significance.
Your content starts here. This section will introduce your statistical analysis project and its significance.
Your content starts here. This section will introduce your statistical analysis project and its significance.
Your content starts here. This section will introduce your statistical analysis project and its significance.

\section{Data Description}

The dataset used in this project was generated
using an automated Python script that reads a
list of websites containing 82 entries
from \texttt{website\_list.json}. The script makes five
HTTP requests to each site and records two
key metrics for each attempt:

\begin{itemize}
  \item \textbf{Page load time (in seconds):}
    The total time required
    for the page to fully load under a web-browser instance.
  \item \textbf{Page size (in kilobytes):}
    The total size of loaded
    webpage including all its resources.
\end{itemize}

These raw measurements were stored in
\texttt{website\_load\_data.json}.
This current data was then processed by the script
\texttt{make\_avg\_csv.py},
which computes the average of the five measurements
of each metric for every website and writes
the summarized results into
\texttt{averaged\_data.csv} in a structured format.
The scripts and datasets used
in this project are available in the following
\href{
  https://github.com/Bubu-Droid/mini-projects-repo/tree/main/lin-reg-sem-1
}{GitHub repository}.
% TODO: add the appendix data given part here

\begin{table}[H]
\centering
\begin{tabular}{||c | c | c | c | c | c||}
\hline
  \textbf{Variable} & \textbf{Mean} & \textbf{Median} & \textbf{Minimum} & \textbf{Maximum} & \textbf{Std. Dev.} \\ [0.5ex]
\hline\hline
  \textbf{Page Size (kb)} & 15427.26 & 6939.39 & 59.6 & 104757.7 & 22067.91 \\
\hline
  \textbf{Load Time (s)} & 8.593405 & 5.755 & 0.3858 & 43.2654 & 7.929519 \\
\hline
\end{tabular}
\caption{Summary statistics of average page size and load time}
\end{table}

\newpage

\section{Model Formulation}

In order to have a better understanding of how the
metrics might be related to each other, we make a
scatter plot by plotting
\textbf{load\_time\_avg} along $Y$-axis
and \textbf{page\_size\_avg} along $X$-axis.

\begin{figure}[H]
  \centering
  \begin{minipage}{0.45\textwidth}
    \centering
    \includegraphics[width=0.9\textwidth,keepaspectratio]{./r-code/default-biv-plot.pdf}
    \caption{Load Time vs. Page Size}
    \label{fig:default-biv}
  \end{minipage}\hfill
  \begin{minipage}{0.45\textwidth}
    \centering
    \includegraphics[width=0.9\textwidth,keepaspectratio]{./r-code/log-biv-plot.pdf}
    \caption{$\log(\text{Load Time})$ vs. $\log(\text{Page Size})$}
    \label{fig:log-biv}
  \end{minipage}
\end{figure}

The relationship between the metrics is not very clear in
\cref{fig:default-biv}. However, after applying a
logarithmic transformation to both variables,
the relationship becomes substantially more linear,
as shown in \cref{fig:log-biv}.

Thus, considering a power-law model of the form
\[
  \text{Load Time}
  = e^{\alpha}\cdot
  \text{Page Size}^{\beta}\cdot e^{\varepsilon}
\]
would be suitable, as this suggests
\[
  \log(\text{Load Time}) = \alpha
  + \beta\log(\text{Page Size}) + \varepsilon,
\]
that is, a linear relationship between
$\log(\text{Load Time})$ and $\log(\text{Page Size})$.

Let $X$ denote the average page size
(in kilobytes) and $Y$ denote the average
page load time (in seconds) for each website.

We will now estimate the parameters
$\alpha$ and $\beta$ using
the method of least squares on the
log-transformed data.

\section{Regression Analysis}

Fitting the linear regression model
\[
  \log(Y) = \alpha + \beta \log(x) + \varepsilon
\]
to the data yields the following estimates:

\begin{table}[H]
  \centering
  \begin{tabular}{||c c c||}

  \hline

    $\hat{\alpha}$ & $\hat{\beta}$ & $R^2$ \\ [0.5ex]

  \hline\hline

   -2.371373 & 0.4780497 & 0.7581765 \\

  \hline

  \end{tabular}
  \caption{Estimated Parameters of the Model}
  \label{table:estimates}
\end{table}

The fitted equation is therefore approximately
\[
  \widehat{\log(Y)} = -2.371 + 0.478\log(x),
\]
with coefficient of determination
$R^2 = 0.758$.

\begin{table}[H]
\centering
\begin{tabular}{||c | c | c | c | c | c||}
\hline
  \textbf{Variable} & \textbf{Mean} & \textbf{Median} & \textbf{Minimum} & \textbf{Maximum} & \textbf{Std. Dev.} \\ [0.5ex]
\hline\hline
  $\log(\text{Page Size})$ & 8.776999 & 8.844928 & 4.087656 & 11.55941 & 1.531387 \\
\hline
  $\log(\text{Load Time})$ & 1.824468 & 1.750042 & -0.9524362 & 3.767353 & 0.8407613 \\
\hline
\end{tabular}
  \caption{Summary statistics of
  $\log(\text{Page Size})$ and $\log(\text{Load Time})$}
  \label{table:summary-stat-log}
\end{table}

So, using values from \cref{table:summary-stat-log}
and \cref{table:estimates}, we get,
\[
  r = \frac{S_{\log(x), \log(Y)}}{S_{\log(x)}S_{\log(Y)}}
  = \hat{\beta}\cdot \frac{S_{\log(x)}}{S_{\log(Y)}}
  = 0.478 \times \frac{1.531}{0.84}
  = 0.87.
\]

Therefore, the sample correlation $r = 0.87$ indicates
a strong positive linear association between the two
variables.

\begin{figure}[H]
  \centering
  \includegraphics[width=0.5\textwidth,keepaspectratio]{./r-code/model-fit.pdf}
  \caption{Scatter plot of $\log(\text{Page Size})$ vs. $\log(\text{Load Time})$ with fitted regression line}
\end{figure}


\section{Interval Estimates}

We now check how the residuals,
$\hat{\varepsilon}_{i} = \log(y_{i}) - \widehat{\log(y_{i})}$,
behave where $\log(y_{i})$ represents a realized value of
$\log(\text{Load Time})$ and $\log(\hat{y_{i}})$ is the fitted
value corresponding to that observation.

\begin{figure}[H]
  \centering
  \begin{minipage}{0.45\textwidth}
    \centering
    \includegraphics[width=0.9\textwidth,keepaspectratio]{./r-code/residual-plot.pdf}
    \caption{Plot of residuals versus fitted values}
    \label{fig:resi-plot}
  \end{minipage}\hfill
  \begin{minipage}{0.45\textwidth}
    \centering
    \includegraphics[width=0.9\textwidth,keepaspectratio]{./r-code/residual-hist.pdf}
    \caption{Histogram of residuals for the log--log regression model}
    \label{fig:resi-hist}
  \end{minipage}
\end{figure}

From \cref{fig:resi-hist}, it can be deduced
that the distribution of residuals roughly follows
a normal distribution with mean $0$.
It is therefore reasonable to assume that
$\varepsilon \sim N(0, \sigma^2)$.
This helps us to estimate confidence and
prediction intervals using quantiles of
$t$-distribution.

The formulas for the confidence intervals are as follows:

\begin{table}[H]
\centering
\begin{tabular}{||c | c||}

\hline
Parameter & Confidence Interval \\ [0.5ex]
\hline\hline
  $\alpha$ 
  & $\hat{\alpha} \pm
  t_{\gamma/2}(n-2)s\sqrt{
    \frac{1}{n} + \frac{\overline{\log(x)}^2}{S_{\log(x)\log(x)}}}$ \\
\hline
  $\beta$ & $\hat{\beta} \pm
  t_{\gamma/2}(n-2) \frac{s}{\sqrt{S_{\log(x)\log(x)}}}$ \\
\hline

\end{tabular}
\caption{$100(1-\gamma)\%$ confidence intervals for model parameters}
\end{table}

We now list the formulas for the $100(1-\gamma)\%$
confidence and prediction intervals for
$\mathbb E[\log(Y)]$ and $\log(Y)$, respectively,
given a new value of $X = x$.

\begin{table}[H]
\centering
\begin{tabular}{||c | c||}

\hline
Quantity & Confidence Interval \\ [0.5ex]
\hline\hline
  $\mathbb E[\log(Y)]$ 
  & $(\hat{\alpha}
  + \hat{\beta}\log(x)) \pm t_{\gamma/2}(n-2)s
  \sqrt{\frac{1}{n}+
  \frac{(\log(x)-\overline{\log(x)})^2}{S_{\log(x)\log(x)}}}$ \\
\hline

\end{tabular}
\caption{$100(1-\gamma)\%$ confidence interval for
  $E[\log(Y)]$ given a new value of $X=x$}
\end{table}

\begin{table}[H]
\centering
\begin{tabular}{||c | c||}

\hline
Quantity & Prediction Interval \\ [0.5ex]
\hline\hline
  $\log(Y)$ 
  & $(\hat{\alpha} + \hat{\beta}\log(x)) \pm
  t_{\gamma/2}(n-2)s\sqrt{1+\frac{1}{n}
  +\frac{(\log(x)-\overline{\log(x)})^2}{S_{\log(x)\log(x)}}}$ \\
\hline

\end{tabular}
\caption{$100(1-\gamma)\%$ prediction interval for
  $\log(Y)$ given a new value of $X=x$}
\end{table}

Substituting dataset values into these formulas
gives the realized upper and lower bounds of the $95\%$
confidence and prediction intervals.

\begin{table}[H]
\centering
\begin{tabular}{||c | c | c||}

\hline
Parameter & Lower Limit & Upper Limit \\ [0.5ex]
\hline\hline
$\alpha$ & -2.906478 & -1.836269 \\
\hline
$\beta$ & 0.4179795 & 0.5381199 \\
\hline

\end{tabular}
\caption{$95\%$ confidence intervals for the regression parameters}
\end{table}

\begin{table}[H]
\centering
\begin{tabular}{||c | c||}

\hline
Quantity & Confidence Interval \\ [0.5ex]
\hline\hline
  $\mathbb E[\log(Y)]$
  & $(-2.3714 + 0.4780 \cdot \log(x))
  \pm 0.8279 \cdot \sqrt{0.0122
  + \frac{(\log(x)-8.7770)^2}{189.9567}}$ \\
\hline

\end{tabular}
  \caption{$95\%$ confidence interval for $\mathbb E[\log(Y)]$
  given a new value of $X=x$}
\end{table}

\begin{table}[H]
\centering
\begin{tabular}{||c | c||}

\hline
Quantity & Prediction Interval \\ [0.5ex]
\hline\hline
  $\log(Y)$ & $(-2.3714 + 0.4780 \cdot \log(x))
  \pm 0.8279 \cdot \sqrt{1.0122
  + \frac{(\log(x)-8.7770)^2}{189.9567}}$ \\
\hline

\end{tabular}
  \caption{$95\%$ prediction interval for $\log(Y)$
  given a new value of $X=x$}
\end{table}


With this, we conclude our interval estimate analysis.


\end{document}
