\documentclass[12pt]{article}
\usepackage{statproj}

\title{Statistical Analysis of the Relationship Between Website Page Size and Load Time}
\author{Avigyan Chakraborty (BS2516)}
\date{November, 2025}

\begin{document}

% First page with title, name, date, and abstract
\maketitlepage{\thetitle}{\theauthor}{\thedate}{
% Your abstract content will go here
This is where you'll write your abstract content later. You can replace this text with your actual abstract when you're ready. The abstract should provide a brief overview of your statistics assignment, including the main objectives, methods, and key findings.

You can write multiple paragraphs here. The formatting will handle the spacing and make it look professional and well-organized for your statistics assignment.
}

% Second page with table of contents (clickable links and improved spacing)
\makecontents

% Third page - start your project here
\section{Introduction}
Your content starts here. This section will introduce your statistical analysis project and its significance.
Your content starts here. This section will introduce your statistical analysis project and its significance.
Your content starts here. This section will introduce your statistical analysis project and its significance.
Your content starts here. This section will introduce your statistical analysis project and its significance.
Your content starts here. This section will introduce your statistical analysis project and its significance.
Your content starts here. This section will introduce your statistical analysis project and its significance.
Your content starts here. This section will introduce your statistical analysis project and its significance.
Your content starts here. This section will introduce your statistical analysis project and its significance.
Your content starts here. This section will introduce your statistical analysis project and its significance.
Your content starts here. This section will introduce your statistical analysis project and its significance.

\section{Data Description}

The dataset used in this project was generated using
an automated script which reads the list of websites,
containing $82$ entries, from
\texttt{website\_list.json} and makes five HTTP requests
to each site and records two key metrics for each attempt:

\begin{itemize}
  \item \textbf{Page load time (in seconds):}
    The total time required
    for the page to fully load under a web-browser instance.
  \item \textbf{Page size (in kilobytes):}
    The total size of loaded
    webpage including all its resources.
\end{itemize}

These raw measurements were then dumped into
\texttt{website\_load\_data.json}. This data was then piped into
the script \texttt{make\_avg\_csv.py} which calculates
the average of five entries of each metric
for each website and then
writes the result into \texttt{averaged\_data.csv}
in a structured manner. The scripts and datasets used
in this project are available in the following
\href{
  https://github.com/Bubu-Droid/mini-projects-repo/tree/main/lin-reg-sem-1
}{GitHub repository}.
% TODO: add the appendix data given part here

\subsection{Summary Statistics}

\begin{table}[htbp]
\centering
\begin{tabular}{||c | c | c | c | c | c||}
\hline
  \textbf{Variable} & \textbf{Mean} & \textbf{Median} & \textbf{Minimum} & \textbf{Maximum} & \textbf{Std. Dev.} \\ [0.5ex]
\hline\hline
  \textbf{Page Size (kb)} & 15427.26 & 6939.39 & 59.6 & 104757.7 & 22067.91 \\
\hline
  \textbf{Load Time (s)} & 8.593405 & 5.755 & 0.3858 & 43.2654 & 7.929519 \\
\hline
\end{tabular}
\caption{Summary statistics of average page size and load time}
\end{table}

\newpage

\section{Model Formulation}

In order to have a better understanding at how the
metrics might be related to each other, we make a
scatter plot by plotting
\textbf{load\_time\_avg} along $Y$-axis
and \textbf{page\_size\_avg} along $X$-axis.

\begin{figure}[htbp]
  \centering
  \begin{minipage}{0.45\textwidth}
    \centering
    \includegraphics[width=0.9\textwidth,keepaspectratio]{./r-code/default-biv-plot.pdf}
    \caption{Load Time vs. Page Size}
    \label{fig:default-biv}
  \end{minipage}\hfill
  \begin{minipage}{0.45\textwidth}
    \centering
    \includegraphics[width=0.9\textwidth,keepaspectratio]{./r-code/log-biv-plot.pdf}
    \caption{$\log(\text{Load Time})$ vs. $\log(\text{Page Size})$}
    \label{fig:log-biv}
  \end{minipage}
\end{figure}

The relationship between the metrics is not very clear in
\cref{fig:default-biv}. However, after applying a logarithmic
transformation to both, the condition improves substantially
as there's a vivid linear relationship between them as
seen in \cref{fig:log-biv}.

Thus considering a power-law model of the form
\[
  \text{Load Time}
  = e^{\alpha}\cdot
  \text{Page Size}^{\beta}\cdot e^{\varepsilon}
\]
would be suitable as this suggests
\[
  \log(\text{Load Time}) = \alpha
  + \beta\log(\text{Page Size}) + \varepsilon,
\]
that is, a linear relationship between
$\log(\text{Load Time})$ and $\log(\text{Page Size})$.

Let $X$ denote the average page size
(in kilobytes) and $Y$ denote the average
page load time (in seconds) for each website.

We will now estimate the parameters
$\alpha$ and $\beta$ using
the method of least squares on the
log-transformed data.

\section{Regression Analysis}

Fitting the linear regression model
\[
  \log(Y) = \alpha + \beta \log(x) + \varepsilon
\]
to the data yields the following estimates:

\begin{table}[htbp]
  \centering
  \begin{tabular}{||c c c||}

  \hline

    $\hat{\alpha}$ & $\hat{\beta}$ & $R^2$ \\ [0.5ex]

  \hline\hline

   -2.371373 & 0.4780497 & 0.7581765 \\

  \hline

  \end{tabular}
  \caption{Estimated Parameters of the Model}
  \label{table:estimates}
\end{table}

The fitted equation is therefore approximately
\[
  \log(\hat{Y}) = -2.371 + 0.478\,\log(x),
\]
with coefficient of determination
$R^2 = 0.758$.

\begin{table}[htbp]
\centering
\begin{tabular}{||c | c | c | c | c | c||}
\hline
  \textbf{Variable} & \textbf{Mean} & \textbf{Median} & \textbf{Minimum} & \textbf{Maximum} & \textbf{Std. Dev.} \\ [0.5ex]
\hline\hline
  $\log(\text{Page Size})$ & 8.776999 & 8.844928 & 4.087656 & 11.55941 & 1.531387 \\
\hline
  $\log(\text{Load Time})$ & 1.824468 & 1.750042 & -0.9524362 & 3.767353 & 0.8407613 \\
\hline
\end{tabular}
  \caption{Summary statistics of
  $\log(\text{Page Size})$ and $\log(\text{Load Time})$}
  \label{table:summary-stat-log}
\end{table}

So, using values from \cref{table:summary-stat-log}
and \cref{table:estimates}, we get,
\[
  r = \frac{S_{x, Y}}{S_xS_y}
  = \hat{\beta}\cdot \frac{S_x}{S_Y}
  = 0.478 \times \frac{1.531}{0.84}
  = 0.87.
\]

Therefore the sample correlation $r = 0.87$ indicates
there is a strong positive linear association between the two
variables.

\begin{figure}[H]
  \centering
  \includegraphics[width=0.6\textwidth,keepaspectratio]{./r-code/model-fit.pdf}
  \caption{Scatter plot of $\log(\text{Page Size})$ vs. $\log(\text{Load Time})$ with fitted regression line}
\end{figure}


\section{Interval Estimates}



\end{document}
