% █▀▀ █▀█ █▄░█ █▀▀ █▀█ ▄▀█ ▀█▀ █░█ █░░ ▄▀█ ▀█▀ █ █▀█ █▄░█ █▀ █
% █▄▄ █▄█ █░▀█ █▄█ █▀▄ █▀█ ░█░ █▄█ █▄▄ █▀█ ░█░ █ █▄█ █░▀█ ▄█ ▄
%
%
%
% You've found an easter egg. Here's your gift:
% https://youtu.be/hB7CDrVnNCs?si=3STkz4Iqvh3KEZEJ
%
%
%
%
\documentclass[12pt]{article}
\usepackage{statproj}

\title{Statistical Analysis of the Relationship Between Website Page Size and Load Time}
\author{Avigyan Chakraborty (BS2516)}
\date{November, 2025}

\begin{document}

% \doublespacing % Oh hell naw, this looks hideous

\maketitlepage{\thetitle}{\theauthor}{\thedate}{
This study investigates the relationship between website
page size and load time using data collected from the
official websites of Indian colleges. For each website,
multiple measurements of page size and load time were taken
under identical network conditions to minimize random
variation due to bandwidth fluctuations. Exploratory
analysis revealed a positive but non-linear association
between the two variables. After applying logarithmic
transformations to both, the relationship became
approximately linear, motivating the use of a power-law
(log--log) regression model.

The fitted model
explained about $75.8\%$ of the variability in load time and
showed a strong positive correlation ($r = 0.87$).
Residual analysis indicated approximate normality,
supporting the validity of inference based on
$t$-distribution. When tested on new websites, all observed
values fell within their respective $95\%$ prediction
intervals, confirming the model's reliability within its
domain. The findings highlight that while page size is a key
determinant of load time for minimalistic websites, other
factors such as caching, server configuration, and use of
CDNs play a major role in more complex, resource-intensive
sites.
}


\section{Introduction}
In the modern web ecosystem, the loading speed of a website
plays a crucial role in determining user experience and
overall accessibility. Websites that take longer to load
often experience higher bounce rates and lower engagement,
making load time an important performance metric. Among the
various factors influencing how fast a webpage loads, the
total size of the page and its associated resources is one
of the most significant.

The objective of this study is to examine the relationship
between the size of a webpage and its corresponding
load time. To achieve this, data were collected from
the official websites of various Indian colleges. The data
include multiple load-time measurements for each site,
allowing us to eliminate the noise due to network
fluctuations and derive
a stable estimate of load duration and total
page size by considering the average.

An initial visual inspection of the raw data suggested a
positive but non-linear association between the two
variables. After applying logarithmic transformations to
both metrics, the relationship became approximately linear,
motivating the use of a log--log regression model.
The subsequent
analysis explores this relationship in detail, estimates the
model parameters, and assesses the fit using confidence and
prediction intervals.

\section{Data Description}

The dataset used in this project was generated
using an automated Python script that reads a
list of 82 websites containing
from \texttt{website\_list.json}. The script makes five
HTTP requests to each site and records two
key metrics for each attempt:

\begin{itemize}
  \item \textbf{Page Load Time (in seconds):}
    The total time required
    for the page to fully load under a web-browser instance.
  \item \textbf{Page Size (in kilobytes):}
    The total size of loaded
    webpage including, all its resources.
\end{itemize}

These raw measurements were stored in
\texttt{website\_load\_data.json}.
This data was then processed by the script
\texttt{make\_avg\_csv.py},
which computes the average of the five measurements
of each metric for every website and writes
the summarized results to
\texttt{averaged\_data.csv} in a structured format.
Considering the average helps us to
eliminate the noises created
due to network fluctuations.
The scripts and datasets used
in this project are available in the following
\href{
  https://github.com/Bubu-Droid/mini-projects-repo/tree/main/lin-reg-sem-1
}{GitHub repository}.

\begin{table}[H]
\centering
\begin{tabular}{||c | c | c | c | c | c||}
\hline
  \textbf{Variable} & \textbf{Mean} & \textbf{Median} & \textbf{Minimum} & \textbf{Maximum} & \textbf{Std. Dev.} \\ [0.5ex]
\hline\hline
  \textbf{Page Size (kb)} & 15427.26 & 6939.39 & 59.6 & 104757.7 & 22067.91 \\
\hline
  \textbf{Load Time (s)} & 8.593405 & 5.755 & 0.3858 & 43.2654 & 7.929519 \\
\hline
\end{tabular}
\caption{Summary statistics of average Page Size and Load Time.}
\end{table}


\section{Model Formulation}

To gain a better understanding of how the
metrics might be related, we make a
scatter plot by plotting
\textbf{load\_time\_avg} along the $Y$-axis
and \textbf{page\_size\_avg} along the $X$-axis.

\begin{figure}[H]
  \centering
  \begin{minipage}{0.45\textwidth}
    \centering
    \includegraphics[width=0.9\textwidth,keepaspectratio]{./r-code/default-biv-plot.pdf}
    \caption{Load Time vs. Page Size.}
    \label{fig:default-biv}
  \end{minipage}\hfill
  \begin{minipage}{0.45\textwidth}
    \centering
    \includegraphics[width=0.9\textwidth,keepaspectratio]{./r-code/log-biv-plot.pdf}
    \caption{$\log(\text{Load Time})$ vs. $\log(\text{Page Size})$.}
    \label{fig:log-biv}
  \end{minipage}
\end{figure}

The relationship between the metrics is not very clear in
\cref{fig:default-biv}. However, after applying a
logarithmic transformation to both variables,
the relationship becomes substantially more linear,
as shown in \cref{fig:log-biv}.

Thus, considering a power-law model of the form
\[
  \text{Load Time}
  = e^{\alpha}\cdot
  \text{Page Size}^{\beta}\cdot e^{\varepsilon}
\]
would be suitable, as this suggests
\[
  \log(\text{Load Time}) = \alpha
  + \beta\log(\text{Page Size}) + \varepsilon,
\]
that is, a linear relationship between
$\log(\text{Load Time})$ and $\log(\text{Page Size})$.

Let $X$ denote the average Page Size
(in kilobytes) and $Y$ denote the average
page Load Time (in seconds) for each website.

We now estimate the parameters
$\hat{\alpha}$ and $\hat{\beta}$ using
the method of least squares on the
log-transformed data.

\section{Regression Analysis}

Fitting the linear regression model
\[
  \log(Y) = \alpha + \beta \log(x) + \varepsilon
\]
to the data yields the following estimates:

\begin{table}[H]
  \centering
  \begin{tabular}{||c c c||}

  \hline

    $\hat{\alpha}$ & $\hat{\beta}$ & $R^2$ \\ [0.5ex]

  \hline\hline

   -2.371373 & 0.4780497 & 0.7581765 \\

  \hline

  \end{tabular}
  \caption{Estimated parameters of the model.}
  \label{table:estimates}
\end{table}

The fitted equation is, therefore, approximately
\[
  \widehat{\log(Y)} = -2.371 + 0.478\log(x),
\]
with coefficient of determination
$R^2 = 0.758$.

\begin{table}[H]
\centering
\begin{tabular}{||c | c | c | c | c | c||}
\hline
  \textbf{Variable} & \textbf{Mean} & \textbf{Median} & \textbf{Minimum} & \textbf{Maximum} & \textbf{Std. Dev.} \\ [0.5ex]
\hline\hline
  $\log(\text{Page Size})$ & 8.776999 & 8.844928 & 4.087656 & 11.55941 & 1.531387 \\
\hline
  $\log(\text{Load Time})$ & 1.824468 & 1.750042 & -0.9524362 & 3.767353 & 0.8407613 \\
\hline
\end{tabular}
  \caption{Summary statistics of
  $\log(\text{Page Size})$ and $\log(\text{Load Time})$.}
  \label{table:summary-stat-log}
\end{table}

So, using values from \cref{table:summary-stat-log}
and \cref{table:estimates}, we get,
\[
  r = \frac{S_{\log(x), \log(Y)}}{S_{\log(x)}S_{\log(Y)}}
  = \hat{\beta}\cdot \frac{S_{\log(x)}}{S_{\log(Y)}}
  = 0.478 \times \frac{1.531}{0.84}
  = 0.87.
\]

Therefore, the sample correlation $r = 0.87$ indicates
a strong positive linear association between the two
variables.

\begin{figure}[H]
  \centering
  \includegraphics[width=0.5\textwidth,keepaspectratio]{./r-code/model-fit.pdf}
  \caption{Scatter plot of $\log(\text{Page Size})$
  vs. $\log(\text{Load Time})$ with fitted regression line.}
\end{figure}


\section{Interval Estimates}

We now check how the residuals
$\hat{\varepsilon}_{i} = \log(y_{i}) - \widehat{\log(y_{i})}$
behave, where $\log(y_{i})$ represents a realized value of
$\log(\text{Load Time})$ and $\widehat{\log(y_{i})}$
is the fitted
value corresponding to that observation.

\begin{figure}[H]
  \centering
  \begin{minipage}{0.45\textwidth}
    \centering
    \includegraphics[width=0.9\textwidth,keepaspectratio]{./r-code/residual-plot.pdf}
    \caption{Plot of residuals vs. Observation index.}
  \end{minipage}\hfill
  \begin{minipage}{0.45\textwidth}
    \centering
    \includegraphics[width=0.9\textwidth,keepaspectratio]{./r-code/residual-hist.pdf}
    \caption{Histogram of residuals for the
    log--log regression model.}
    \label{fig:resi-hist}
  \end{minipage}
\end{figure}

From \cref{fig:resi-hist}, it can be deduced
that the distribution of residuals roughly follows
a normal distribution with mean $0$.
It is therefore reasonable to assume that
$\varepsilon \sim N(0, \sigma^2)$.
This helps us to estimate confidence and
prediction intervals using quantiles of
$t$-distribution.

The formulas for the confidence intervals are as follows:

\begin{table}[H]
\centering
\begin{tabular}{||c | c||}

\hline
\textbf{Parameter} & \textbf{Confidence Interval} \\ [0.5ex]
\hline\hline
  $\alpha$ 
  & $\hat{\alpha} \pm
  t_{\gamma/2}(n-2)s\sqrt{
    \frac{1}{n} + \frac{\overline{\log(x)}^2}{S_{\log(x)\log(x)}}}$ \\
\hline
  $\beta$ & $\hat{\beta} \pm
  t_{\gamma/2}(n-2) \frac{s}{\sqrt{S_{\log(x)\log(x)}}}$ \\
\hline

\end{tabular}
\caption{$100(1-\gamma)\%$ confidence intervals
  for model parameters.}
\end{table}

We now list the formulas for the $100(1-\gamma)\%$
confidence and prediction intervals for
$\mathbb E[\log(Y)]$ and $\log(Y)$, respectively,
given a new value of $X = x$.

\begin{table}[H]
\centering
\begin{tabular}{||c | c||}

\hline
\textbf{Quantity} & \textbf{Confidence Interval} \\ [0.5ex]
\hline\hline
  $\mathbb E[\log(Y)]$ 
  & $(\hat{\alpha}
  + \hat{\beta}\log(x)) \pm t_{\gamma/2}(n-2)s
  \sqrt{\frac{1}{n}+
  \frac{(\log(x)-\overline{\log(x)})^2}{S_{\log(x)\log(x)}}}$ \\
\hline

\end{tabular}
\caption{$100(1-\gamma)\%$ confidence interval for
  $E[\log(Y)]$ given a new value of $X=x$.}
\end{table}

\begin{table}[H]
\centering
\begin{tabular}{||c | c||}

\hline
\textbf{Quantity} & \textbf{Prediction Interval} \\ [0.5ex]
\hline\hline
  $\log(Y)$ 
  & $(\hat{\alpha} + \hat{\beta}\log(x)) \pm
  t_{\gamma/2}(n-2)s\sqrt{1+\frac{1}{n}
  +\frac{(\log(x)-\overline{\log(x)})^2}{S_{\log(x)\log(x)}}}$ \\
\hline

\end{tabular}
\caption{$100(1-\gamma)\%$ prediction interval for
  $\log(Y)$ given a new value of $X=x$.}
\end{table}

Substituting dataset values into these formulas
gives the realized upper and lower bounds of the $95\%$
confidence and prediction intervals.

\begin{table}[H]
\centering
\begin{tabular}{||c | c | c||}

\hline
  \textbf{Parameter}
  & \textbf{Lower Limit}
  & \textbf{Upper Limit} \\ [0.5ex]
\hline\hline
$\alpha$ & -2.906478 & -1.836269 \\
\hline
$\beta$ & 0.4179795 & 0.5381199 \\
\hline

\end{tabular}
\caption{$95\%$ confidence intervals for the
  regression parameters.}
\end{table}

\begin{table}[H]
\centering
\begin{tabular}{||c | c||}

\hline
\textbf{Quantity} & \textbf{Confidence Interval} \\ [0.5ex]
\hline\hline
  $\mathbb E[\log(Y)]$
  & $(-2.3714 + 0.4780 \cdot \log(x))
  \pm 0.8279 \cdot \sqrt{0.0122
  + \frac{(\log(x)-8.7770)^2}{189.9567}}$ \\
\hline

\end{tabular}
  \caption{$95\%$ confidence interval for $\mathbb E[\log(Y)]$
  given a new value of $X=x$.}
\end{table}

\begin{table}[H]
\centering
\begin{tabular}{||c | c||}

\hline
  \textbf{Quantity} & \textbf{Prediction Interval} \\ [0.5ex]
\hline\hline
  $\log(Y)$ & $(-2.3714 + 0.4780 \cdot \log(x))
  \pm 0.8279 \cdot \sqrt{1.0122
  + \frac{(\log(x)-8.7770)^2}{189.9567}}$ \\
\hline

\end{tabular}
  \caption{$95\%$ prediction interval for $\log(Y)$
  given a new value of $X=x$.}
\end{table}

We now put the prediction interval to test using
a few new websites as test cases.

\begin{table}[H]
\centering
\begin{tabular}{||c | c | c | c||}

\hline
  \textbf{Website}
  & \textbf{Prediction Interval for $\log(Y)$}
  & \textbf{Actual $\log(y)$}
  & \textbf{Inside interval?} \\ [0.5ex]
\hline\hline
  Amazon & $(0.1462, 1.8253)$ & $0.432$ & Yes \\
\hline
  Project Euler & $(-1.0737, 0.6683)$ & $0.377$ & Yes \\
\hline
  Suckless & $(-0.9869, 0.7490)$ & $0.586$ & Yes \\
\hline

\end{tabular}
\caption{Comparing bounds of prediction interval
  with actual value of $\log(\text{Load Time})$.}
\end{table}


\section{Conclusion}

The analysis demonstrates a strong positive relationship
between average web page size and average load time.
After applying logarithmic transformations to both
variables, the fitted model
\[
  \widehat{\log(Y)} = -2.371 + 0.478 \log(x)
\]
explained about $75.8\%$ of the variation in the
log-transformed load time. The estimated correlation
coefficient of $r = 0.87$ further confirms a strong linear
association between these variables.

Residual analysis showed that the error terms were
approximately normally distributed, validating the use of
$t$-based confidence and prediction intervals. When tested
on new websites, all observed $\log(\text{Load Time})$
values fell within their respective $95\%$ prediction
intervals, indicating that the model performs reliably
within its observed domain.

It should be noted that all data were collected from
websites of Indian colleges, which are generally
minimalistic and have relatively small page sizes compared
to large commercial or media websites. Furthermore, several
external factors influencing page load time, such as, the use
of Content Delivery Networks (CDNs), server response time,
caching mechanisms, client-side rendering, and network
routing were not modeled explicitly. However, since all
measurements were made using the same network and under
similar conditions, much of the variation due to network
speed was effectively smoothed out. As a result, the
regression primarily captures the structural relationship
between page size and load time for lightweight, static
websites. Nevertheless, the model may not generalize well
to resource-intensive, dynamically rendered websites that
depend heavily on JavaScript execution, third-party assets,
or geographically distributed servers.

\appendix
\section{Appendix}

\subsection{Dataset}

\begin{longtable}{||p{6cm} | c | c||}
\caption{Raw Average Data for Websites.} \\
\hline
\textbf{Website} & \textbf{Load Time (s)} & \textbf{Page Size (KB)} \\ [0.5ex]
\hline\hline
\endfirsthead

\multicolumn{3}{c}
{{\bfseries \tablename\ \thetable{} -- continued from previous page}} \\
\hline
\textbf{Website} & \textbf{Load Time (s)} & \textbf{Page Size (KB)} \\ [0.5ex]
\hline\hline
\endhead

\hline
\endfoot

\hline
\endlastfoot

Indian Statistical Institute, Kolkata & 10.3276 & 7066.44 \\
\hline
Indian Statistical Institute, Delhi & 0.3858 & 59.6 \\
\hline
Indian Statistical Institute, Bangalore & 1.0156 & 76.0 \\
\hline
Indian Statistical Institute, Chennai & 1.2878 & 1940.2 \\
\hline
Indian Statistical Institute, Pune & 1.1098 & 2093.1 \\
\hline
Indian Statistical Institute, Tezpur & 7.584 & 14778.2 \\
\hline
Chennai Mathematical Institute & 1.2328 & 62.14 \\
\hline
Tata Institute of Fundamental Research & 9.2808 & 4573.1 \\
\hline
TIFR Centre for Applicable Mathematics & 3.6046 & 333.84 \\
\hline
TIFR Centre for Interdisciplinary Sciences & 8.1344 & 21874.06 \\
\hline
International Centre for Theoretical Sciences & 11.5872 & 26943.2 \\
\hline
Harish-Chandra Research Institute & 4.4566 & 4541.56 \\
\hline
Institute of Mathematical Sciences & 3.5582 & 1548.02 \\
\hline
Raman Research Institute & 2.4424 & 722.84 \\
\hline
Physical Research Laboratory & 9.3362 & 27668.32 \\
\hline
Inter-University Centre for Astronomy and Astrophysics & 5.4912 & 1449.8 \\
\hline
S.N. Bose National Centre for Basic Sciences & 4.975 & 5124.2 \\
\hline
Institute of Physics, Bhubaneswar & 7.7596 & 22306.3 \\
\hline
Indian Institute of Science & 9.3522 & 5671.52 \\
\hline
IISER Pune & 2.9018 & 2656.14 \\
\hline
IISER Kolkata & 4.8636 & 5098.6 \\
\hline
IISER Bhopal & 20.3772 & 53687.0 \\
\hline
IISER Tirupati & 5.0602 & 2817.76 \\
\hline
IISER Berhampur & 15.249 & 33293.1 \\
\hline
IIT Bombay & 1.2544 & 571.04 \\
\hline
IIT Delhi & 5.0844 & 12550.1 \\
\hline
IIT Madras & 3.6296 & 6350.46 \\
\hline
IIT Kanpur & 7.7106 & 9316.92 \\
\hline
IIT Kharagpur & 6.9596 & 6664.0 \\
\hline
IIT Guwahati & 11.506 & 31890.84 \\
\hline
IIT Roorkee & 28.965 & 95601.9 \\
\hline
IIT Hyderabad & 5.4708 & 14873.24 \\
\hline
IIT Indore & 4.6852 & 4374.22 \\
\hline
IIT BHU & 20.5672 & 12158.16 \\
\hline
IIT Gandhinagar & 5.7974 & 5024.42 \\
\hline
IIT Patna & 4.9366 & 2779.06 \\
\hline
IIT Mandi & 17.538 & 9794.48 \\
\hline
IIT Bhubaneswar & 42.9686 & 89554.08 \\
\hline
IIT Tirupati & 4.645 & 1548.1 \\
\hline
IIT Palakkad & 5.7126 & 2446.7 \\
\hline
IIT Dhanbad (ISM) & 5.466 & 8630.1 \\
\hline
IIT Dharwad & 8.859 & 8012.36 \\
\hline
NIT Trichy & 5.1724 & 5413.0 \\
\hline
NIT Surathkal & 5.8766 & 3992.5 \\
\hline
NIT Calicut & 21.2912 & 70576.94 \\
\hline
NIT Durgapur & 3.9468 & 7002.16 \\
\hline
NIT Silchar & 6.43 & 11423.16 \\
\hline
NIT Meghalaya & 14.49 & 33129.52 \\
\hline
NIT Agartala & 14.2918 & 13669.86 \\
\hline
NIT Raipur & 4.2412 & 4623.38 \\
\hline
NIT Kurukshetra & 4.9326 & 2517.9 \\
\hline
NIT Srinagar & 7.5526 & 18134.58 \\
\hline
NIT Arunachal Pradesh & 4.705 & 3800.4 \\
\hline
NIT Nagaland & 6.5974 & 8995.5 \\
\hline
NIT Sikkim & 6.4048 & 17421.0 \\
\hline
NIT Goa & 16.4818 & 44634.9 \\
\hline
NIT Puducherry & 11.6826 & 31309.92 \\
\hline
Delhi University & 5.9378 & 7839.2 \\
\hline
BITS Pilani & 3.8468 & 4183.3 \\
\hline
Jawaharlal Nehru University & 2.7822 & 6876.62 \\
\hline
Banaras Hindu University & 9.7682 & 26132.32 \\
\hline
University of Calcutta & 15.5968 & 9403.62 \\
\hline
Indian Institute of Engineering Science and Technology, Shibpur & 11.9522 & 31924.2 \\
\hline
Anna University & 8.8914 & 5229.74 \\
\hline
Savitribai Phule Pune University & 3.6592 & 6562.66 \\
\hline
Aligarh Muslim University & 5.2848 & 3853.84 \\
\hline
University of Madras & 43.2654 & 104757.7 \\
\hline
Panjab University & 5.1862 & 9035.3 \\
\hline
Central University of Rajasthan & 5.1708 & 1740.1 \\
\hline
Maulana Azad National Institute of Technology (MANIT) & 17.672 & 28977.98 \\
\hline
Visvesvaraya National Institute of Technology (VNIT) & 7.2212 & 11702.72 \\
\hline
Goa University & 4.027 & 3244.68 \\
\hline
Jawaharlal Nehru Centre for Advanced Scientific Research (JNCASR) & 4.8342 & 13388.84 \\
\hline
Saha Institute of Nuclear Physics (SINP) & 3.6876 & 1473.14 \\
\hline
National Centre for Biological Sciences (NCBS) & 8.8122 & 18537.66 \\
\hline
Institute of Plasma Research (IPR) & 12.1688 & 14067.02 \\
\hline
Bhabha Atomic Research Centre (BARC) Training School & 5.6862 & 12703.36 \\
\hline
Institute of Genomics and Integrative Biology (IGIB) & 2.5644 & 1085.5 \\
\hline
Indira Gandhi Centre for Atomic Research (IGCAR) Training School & 27.3082 & 87850.6 \\
\hline
Vellore Institute of Technology (VIT) & 6.0696 & 5411.14 \\
\hline
Jamia Millia Islamia & 5.0288 & 3526.96 \\
\hline
Visva-Bharati University & 1.0108 & 356.8 \\
\end{longtable}


\subsection{R Codes}

\subsubsection{Code for printing summary table}
\inputminted{r}{./r-code/summary-stat.R}

\subsubsection{Code for making the Load Time vs. Page Size bivariate plot}
\inputminted{r}{./r-code/default-biv-plot.R}

\subsubsection{Code for making the log(Load Time) vs. log(Page Size) bivariate plot}
\inputminted{r}{./r-code/log-biv-plot.R}

\subsubsection{Code for fitting the linear regression}
\inputminted{r}{./r-code/model-fit.R}

\subsubsection{Code for plotting residuals}
\inputminted{r}{./r-code/residual-plot.R}

\subsubsection{Code for making a histogram for the residuals}
\inputminted{r}{./r-code/residual-hist.R}

\subsection{Python Codes}

The Python scripts can be found at the following
\href{
  https://github.com/Bubu-Droid/mini-projects-repo/tree/main/lin-reg-sem-1/data
}{GitHub repository}.

\end{document}
